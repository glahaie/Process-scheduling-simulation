\documentclass[letterpaper,11pt]{letter}
%\documentclass[letterpaper,12pt]{scrartcl}
\renewcommand{\familydefault}{\sfdefault}
\usepackage[utf8x]{inputenc}
\usepackage[margin=2cm]{geometry}
\usepackage{amsmath}
\usepackage{amssymb}
\usepackage{amsfonts}
\usepackage{tikz}
\usepackage{tikz-qtree}
\usepackage{cancel}
\usetikzlibrary{decorations.markings}

\title{INF5151 TP1}
\author{Guillaume Lahaie}
\date{Remise: 14 février 2013}

\pdfinfo{%
  /Title    (INF5151 TP1)
  /Author   (Guillaume Lahaie)
}

\begin{document}
\tikzset{every tree node/.style={minimum width=2em,draw,circle},
         blank/.style={draw=none},
         edge from parent/.style=
         {draw, edge from parent path={(\tikzparentnode) -- (\tikzchildnode)}},
         level distance=1.5cm}
\begin{center}{\Large{\bf Guillaume Lahaie\\
LAHG04077707\\
TP1\\
INF5151\\
Date de remise: 14 février 2013}}\\
\end{center}

Système de vente et d'achat

{\Large 1. Déterminer les use-cases de ce système. Pour chacun, proposer une description textuelle.}

Voici les uses cases:



\begin{tabular}{|l|p{0.80\textwidth}|}
  \hline
{\bf Titre:} & Consulter des catalogues\\
  \hline
{\bf Type:} & Use case générale\\
\hline
{\bf Description:} & L'acteur consulte des catalogue du système de vente et d'achat.\\
\hline
{\bf Acteurs:} & Un client\\
\hline
{\bf Préconditions:} & Aucune\\
\hline
{\bf Postconditions:} & Le catalogue est disponible au client\\
\hline
\end{tabular}

La client demande au système de consulter un catalogue. Le sytème demande alors au client s'il est membre du membre. S'il est membre, le client entre 
alors ses informations de connexion. Il a alors accès à tous les catalogues du système.

Le client non-membre du système peut alors consulter une sélection restreinte des catalogues du système.

\begin{center}
\line(1,0){250}
\end{center}

\begin{tabular}{|l|p{0.80\textwidth}|}
  \hline
{\bf Titre:} & Effectuer une recherche dans le système\\
  \hline
{\bf Type:} & Use case générale\\
\hline
{\bf Description:} & L'acteur soumet une requête de recherche au système, et le système lui fournit une liste des résultats.\\
\hline
{\bf Acteurs:} & Membre\\
\hline
{\bf Préconditions:} & L'acteur est un membre du système\\
\hline
{\bf Postconditions:} & Le système fournit la liste des résutats.\\
\hline
\end{tabular}

Un membre s'authentifie auprès du système, il soumet une requête de recherche au système. Le système vérifie alors si le membre a assez de crédits pour effectuer
la recherche. Si oui, le système effectue la recherche et founit au membre la liste des résultats. Il retire aussi les crédits nécessaires à la recherche au compte
du membre.

Si le membre n'a pas assez de crédit pour effectuer la recherche, le système lui envoi un message indiquant le manque de crédits.

\begin{center}
\line(1,0){250}
\end{center}

\begin{tabular}{|l|p{0.80\textwidth}|}
  \hline
{\bf Titre:} & S'inscrire en tant que membre\\
  \hline
{\bf Type:} & Use case générale\\
\hline
{\bf Description:} & L'acteur remplie une demande pour s'inscrire en tant que membre\\
\hline
{\bf Acteurs:} & Non-membre\\
\hline
{\bf Préconditions:} & L'acteur n'est pas déjà membre du système.\\
\hline
{\bf Postconditions:} & Une demande d'inscription est envoyée à l'administrateur du système.\\
\hline
\end{tabular}

L'acteur demande au système de s'inscrire comme membre. Le système demande à l'acteur de lui fournir certaines informations. Il demande aussi à l'acteur 
de fournir un paiement pour un abonnement d'un an. Le système envoie alors la demande à l'administrateur système.

\begin{center}
\line(1,0){250}
\end{center}

\begin{tabular}{|l|p{0.80\textwidth}|}
  \hline
{\bf Titre:} & Ajouter un item au système de vente\\
  \hline
{\bf Type:} & Use case générale\\
\hline
{\bf Description:} & L'acteur ajoute qu'il désire vendre au système de vente.\\
\hline
{\bf Acteurs:} & Membre\\
\hline
{\bf Préconditions:} & L'acteur est membre du système.\\
\hline
{\bf Postconditions:} & L'item est ajouté au système de vente.\\
\hline
\end{tabular}

Le membre s'authentifie auprès du système. Il demande alors au système de d'ajouter un item en vente. Le système vérifie si le membre a assez de 
crédits pour faire cette action. Si ce n'est pas le cas, il en informe le membre. Si le membre a assez de crédits, le système demande alors des informations sur l'item: prix demandé, le catalogue où il devrait apparaître, description de l'objet. Le système enregistre les informations, ajoute l'item au catalogue demandé et débite le compte de crédits du membre.

\begin{center}
\line(1,0){250}
\end{center}

\begin{tabular}{|l|p{0.80\textwidth}|}
  \hline
{\bf Titre:} & Retirer un item au système de vente\\
  \hline
{\bf Type:} & Use case générale\\
\hline
{\bf Description:} & L'acteur retire un item qu'il a mis en vente sur le système.\\
\hline
{\bf Acteurs:} & Membre\\
\hline
{\bf Préconditions:} & L'acteur est membre du système et l'item à retirer est présent dans le système.\\
\hline
{\bf Postconditions:} & L'item est rétiré au système de vente.\\
\hline
\end{tabular}

Le membre s'authentifie auprès du système. Il demande alors au système de lui fournir une liste des items qu'il a présentement en vente. Il choisit
l'item à retiré. Le système lui demande de confirmer le retrait et de fournir une raison du retrait, si le membre le désire.

\begin{center}
\line(1,0){250}
\end{center}

\begin{tabular}{|l|p{0.80\textwidth}|}
  \hline
{\bf Titre:} & Envoyer un message à un membre\\
  \hline
{\bf Type:} & Use case générale\\
\hline
{\bf Description:} & L'acteur envoie un message à un autre membre du système.\\
\hline
{\bf Acteurs:} & Membre\\
\hline
{\bf Préconditions:} & L'acteur est membre du système.\\
\hline
{\bf Postconditions:} & Un message est envoyé au membre choisi.\\
\hline
\end{tabular}

Le membre s'authentifie auprès du système. Il demande au système d'envoyer un message à un membre du système. Le système demande à l'acteur d'entrer 
l'identifiant du membre, et le message qu'il désire envoyé. Le système vérifie alors si l'acteur a assez de crédit pour effectuer cette action. Dans
le cas positif, le message est envoyé et le système confirme l'envoie à l'acteur. Dans le cas négatif, le système avise l'acteur qu'il n'a pas assez
de crédit pour envoyer le message.

\begin{center}
\line(1,0){250}
\end{center}

\begin{tabular}{|l|p{0.80\textwidth}|}
  \hline
{\bf Titre:} & Lire les messages reçus\\
  \hline
{\bf Type:} & Use case générale\\
\hline
{\bf Description:} & L'acteur prend connaissance des messages reçus.\\
\hline
{\bf Acteurs:} & Membre\\
\hline
{\bf Préconditions:} & L'acteur est membre du système.\\
\hline
{\bf Postconditions:} & L'acteur a pris connaissance des messages. \\
\hline
\end{tabular}

Le membre s'authentifie auprès du système. Il demande au système de lui fournir les messages reçus. Le système retourne au membre la liste des nouveaux messages.
L'acteur prend connaissance des messages, ensuite il peut effacer les messages, ou le conserver.

\begin{center}
\line(1,0){250}
\end{center}

\begin{tabular}{|l|p{0.80\textwidth}|}
  \hline
{\bf Titre:} & Consulter les crédits restants\\
  \hline
{\bf Type:} & Use case générale\\
\hline
{\bf Description:} & L'acteur consulte son nombre de crédits restants dans le système.\\
\hline
{\bf Acteurs:} & Membre\\
\hline
{\bf Préconditions:} & L'acteur est membre du système.\\
\hline
{\bf Postconditions:} & Le système fournit à l'acteur le nombre de crédits restants.\\
\hline
\end{tabular}

Le membre s'authentifie auprès du système. Il demande au système de lui fournir le nombre de crédits restants. Le système fournit à l'utilisateur cette information.

\begin{center}
\line(1,0){250}
\end{center}

\begin{tabular}{|l|p{0.80\textwidth}|}
  \hline
{\bf Titre:} & Créer un compte\\
  \hline
{\bf Type:} & Use case générale\\
\hline
{\bf Description:} & L'acteur crée un compte dans le système\\
\hline
{\bf Acteurs:} & Administrateur du système.\\
\hline
{\bf Préconditions:} & Le compte à créer est associé à une demande\\
\hline
{\bf Postconditions:} & Le compte est créé.\\
\hline
\end{tabular}

L'administrateur s'authentifie auprès du système. Il demande au système de créer un nouveau compte. Le système affiche les informations du compte 
qui sera créé. L'administrateur vérifie les informations. Il vérifie aussi que le paiement est bien effectué. L'administateur demande au système de
créer le compte. Le système crée le compte, ajoute les crédits et envoi un message au nouveau membre pour confirmer la création.

\begin{center}
\line(1,0){250}
\end{center}

\begin{tabular}{|l|p{0.80\textwidth}|}
  \hline
{\bf Titre:} & Demande de modifier les informations d'un compte\\
  \hline
{\bf Type:} & Use case générale\\
\hline
{\bf Description:} & L'acteur demande au système de modifier certaines informations de son compte.\\
\hline
{\bf Acteurs:} & Membre\\
\hline
{\bf Préconditions:} & ?\\
\hline
{\bf Postconditions:} & Une demande de modification est envoyée à l'administrateur.\\
\hline
\end{tabular}

Le membre s'authentifie auprès du système. Il demande au système de modifier les informations de son compte. Le système demande au membre de confirmer les
changements à effectuer. Le système envoie la demande de modification à l'administrateur du système.

\begin{center}
\line(1,0){250}
\end{center}

\begin{tabular}{|l|p{0.80\textwidth}|}
  \hline
{\bf Titre:} & Modifier les informations d'un compte\\
  \hline
{\bf Type:} & Use case générale\\
\hline
{\bf Description:} & L'acteur effectue les modifications demandés à un compte.\\
\hline
{\bf Acteurs:} & Administateur du système.\\
\hline
{\bf Préconditions:} & Réception d'une demande de modification d'un compte\\
\hline
{\bf Postconditions:} & Le compte est modifié.\\
\hline
\end{tabular}

L'administrateur s'authentifie auprès du système. Il demande au système de lui fournir le compte à être modifié. L'administrateur effectue les
modifications et enregistre les changements. Le système envoie un message de confirmation au membre.

\begin{center}
\line(1,0){250}
\end{center}

\begin{tabular}{|l|p{0.80\textwidth}|}
  \hline
{\bf Titre:} & Renouveller un abonnement\\
  \hline
{\bf Type:} & Use case générale\\
\hline
{\bf Description:} & L'acteur effectue les modifications demandés à un compte.\\
\hline
{\bf Acteurs:} & Administateur du système.\\
\hline
{\bf Préconditions:} & Réception d'une demande de modification d'un compte\\
\hline
{\bf Postconditions:} & Le compte est modifié.\\
\hline
\end{tabular}

L'administrateur s'authentifie auprès du système. Il demande au système de lui fournir le compte à être modifié. L'administrateur effectue les
modifications et enregistre les changements. Le système envoie un message de confirmation au membre.


2. Proposer un diagramme de Use-Cases de haut niveau, faisant apparaître
clairement, acteurs et Use-Cases.
3. Proposer un diagramme de Use-Cases plus détaillé, faisant apparaître des
relations USE et EXTEND.
4. À la vue des Use-Cases proposés, quels sont les concepts importants à
retenir pour un futur modèle de classes.



\end{document}
